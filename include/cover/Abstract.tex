Evaluation of Technical Debt in the context of GUI testware: an exploratory case study in industry\\
%A Subtitle that can be Very Much Longer if Necessary\\
Marcello Steiner\\
Department of Software Engineering\\
Chalmers University of Technology\\

\thispagestyle{plain}			% Supress header 
\section*{Abstract}
Technical debt (TD) is a concept used to describe
a sub-optimal solution of a software artifact that hinders its maintainability.
The difference in cost between maintaining a solution with sub-optimal decisions and one without is called interest. Academia has identified TD in all types of software artifacts,
from architectural design to automated tests (Testware). However,
research into testware technical debt (TTD) is limited and
primarily focused on testing on lower level of system abstraction,
i.e.\ unit and integration tests, leaving a need for TTD research
on GUI-based testing. This Thesis aims to explore this gap in knowledge through an
industrial case study at a Swedish avionics software house. The study used a high level of data triangulation; the sources include four projects, expert interviews, semi-automated document analysis and automatic metric analysis. The results of the study highlight a partial match between the available TD knowledge and TTD. Moreover, two novel TTD items have been identified. The implications of these results are that development best practices are required
for GUI-based testware to minimize TD accretion.

% KEYWORDS (MAXIMUM 10 WORDS)
\vfill
Keywords: Technical Debt, User Interface testing, testing, UFT, Interest, Case study

\newpage				% Create empty back of side
\thispagestyle{empty}
\mbox{}