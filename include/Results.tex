\chapter{Results} \label{study_results}

This Chapter describes all the results gathered throughout the study in a per RQ basis. This approach increase the traceability of the findings within the report as suggested by Runeson et al. \cite{case_study_software_engineering} (Item 20 in Appendix \ref{checklist_for_case_studies}).

As a final note, due to confidentiality issues it was not possible to report the raw data. For this reason, as briefly explained in the previous Chapter, all the sensitive information has been eliminated and anonymized.

\section{Research Question 1}

The aim of this research question is to unveil to what extent the available body of knowledge targeting Code decay is applicable to User Interface test scripts. Intuitively, UI tests suffer from these problems. In fact, they consist of a series of pre-recorded events that lead to boolean assertions regulated by control flow constructs (e.g. If-Else).



This fact exposes such code bases to well known problems \cite{code_smell_definition,domain_specific_code_smells} that impact maintainability. For instance, a series of deeply nested If-else constructs makes difficult to backtrack the data flow and hence result in error-prone maintenance activities.

\section{Research Question 2}

The scope of User Interface tests is to validate the state of the User Interface after a predefined series of actions occurred. However, this entails a different use for the constructs compared to conventional source code and possibly new ones specific for these use cases. For this reason the second RQ will inspect these unique items.

However, keeping into account the differences between property based testing and image recognition testing is important. The former uses property or meta-properties of UI components to identify them and simulate user events over them. The latter, instead, uses image recognition algorithms to identify such widgets. This different identification method imposes different approached when creating the test suite that are analyzed separately in the following sections.

\todo{Talk about the coordinates based approach}

\subsection{Property-based problems}

\subsection{Coordinate-based problems}

\subsection{Image-recognition problems}


\section{Research Question 3}

Finally, the purpose of this RQ is to properly characterize TD items identified in the previous sections. The interest, in fact, divides TD items that must be solved immediately from ones that allow a strategical management of Technical Debt itself. 

