\chapter{Validity Threats} \label{validity_threats}

Validity threats are the factors that can decrease the trustworthiness of the study due to a subjective bias introduced by researchers. In order to keep this phenomenon at bay such factors must be continuously considered throughout the study.

Academia proposes different approaches to categorize these items, but for clarity's and consistency's sake I decided to follow the guidelines expressed by \cite{case_study_guide,case_study_software_engineering}. In fact, the content of these studies has been used elsewhere within this study and their relevance within the field is top notch. Furthermore, I also considered the fine grained categorization of validity threats proposed by Trochim and Donnelly \cite{validity_threats}.

\section{Construct Validity}
Construct validity targets the soundness of the relationship between operational measures and the purpose/RQs of the study. There are many factors that can accrue this threat and I want to address them separately.

\todo{Define what a construct is}

    \textit{Hypothesis guessing} happens when the subjects under study guess the purpose of the study and consciously or not alter their behavior. I considered this factor while performing the semi-structured interviews. In fact, even though its relevance is limited withing the software engineering field, it still can influence the answers of the interviewees. My approach to this problem was to avoid disclosing as much as possible information about the study.
    
    \textit{Evaluation apprehension} considers that when under stress subjects tend to act differently that usual. To leverage this problem during the interviews I stated at the beginning of each one of them that only me has access to the raw data and anything included in the final report or the thesis' database is anonymized.
    
    Another factor related with the human nature is the \textit{researcher expectancies and bias}. In fact, researchers tend to interpret results in a way that favorites their ideas and hence introduce bias in the resutls. In order to leverage this issue I extensively used coding when analyzing qualitative data and achieved a high level of data triangulation. Furthermore, I had the report draft continuously revised by key-informants \cite{case_study_guide}.
    
    A different kind of threat that considers the intrinsic quality of the construct is called \textit{poor construct definition}. In fact, if this is too narrow or too broad the results of the whole study are not reliable. Exploratory studies like this one marginally suffer this kind of problem since the construct definition is gradually built withing the study itself.
    
    \textit{Mono method bias} happens when a single method of measurement is used and is applicable to the quantitative data that has been gathered in this study. Unluckily, given the time constraints that the Thesis impose i had to rely on a single method. However, given that the sole purpose of the analysis of the repositories is to validate the theory proposal crafted I believe that this is acceptable. Furthermore, I also used semi-structured interview for the same purpose.

\section{External Validity}
This validity threats consider if the findings of the study apply to other subjects/cases. However, for case studies the purpose is to enable analytical generalization in order to extend the results to cases with similar settings \cite{case_study_software_engineering}. Yin \cite{case_study_guide} states that this particular threat must be addressed when designing the study.

In this thesis, which follows the single case study approach, the cited literature suggests to make an extensive use of the theory in order to build robust foundations that, in turn, increase generalizability.

As it is possible to see from the Related Work section, I followed a systematic approach to obtain the most relevant and recent papers on the topic.

\section{Internal Validity}
Internal validity estimates the extent to which a casual relationships between variables is trustworthy and is achieved by removing or minimizing the bias. Although this threat type is more applicable to experiments or explanatory studies, there is one factor of interest for this thesis.

\textit{Selection bias} is relevant when sampling resources that are used to extract or validate the data. Therefore, it plays a central role when selecting assets for the theory validation phase. In particular, in order to obtain a sound validation it requires people with high expertise on the topic and that were involved in the development of the test suites that were mined as part of the study. For this reason the interviewees chosen are the five top contributors to such repositories.


\section{Reliability}