\chapter{Validity Threats} \label{validity_threats}

Validity threats are the factors that can decrease the trustworthiness of the study due to a subjective bias introduced by researchers and to keep this phenomenon at bay such factors must be continuously considered throughout the execution.

Academia proposes different approaches to categorize the different parts that concur in mining a study reliability, but for clarity's and consistency's sake the guidelines expressed by \cite{case_study_guide,case_study_software_engineering} will be followed. In fact, the content of these studies has been used elsewhere within this study and their relevance within the field is top notch. Furthermore, the fine grained categorization of validity threats proposed by Trochim and Donnelly \cite{validity_threats} has also been used to list relevant aspects.

\section{Construct Validity}
Construct validity targets the soundness of the relationship between operational measures and the RQs of the study. There are many factors that can accrue this threat and are addressed separately in the following paragraphs.

    \textit{Hypothesis guessing} happens when the subjects under study guess the purpose of the study and consciously or not alter their behavior accordingly. The preparation of semi-structured interviews consider this aspect. In fact, even though its relevance is limited within the software engineering field, it still can influence the answers of the interviewees. The approach selected for leveraging this phenomenon consisted in not revealing details about the study and, hence, not giving the subjects the basic information that might have triggered their curiosity. Precisely, informal interviews have been conducted with other tester that were not selected for either metrics' validations or finding validations interviews.

    \textit{Evaluation apprehension} considers unusual behavior induced to study subjects by the stress of knowing of being studied. They tend to act differently that usual. To leverage this problem, Interviews are kept absolutely confidential, which was stated during the interviews and also printed before the question list. Moreover, subjects were aware that only the author of this study has access to the raw data to minimize the stress of being pinpointed due to the small number of intervieews.

    Another factor related with the human nature is the \textit{researcher expectancies and bias}. In fact, researchers tend to interpret results in a way that favorites their ideas and hence introduce bias in the results. In order to leverage this issue this study used coding extensively when analyzing qualitative data. Moreover, achieved a high level of data triangulation by including several data sources. Furthermore, the report draft was reviewed continuously revised by key-informants, as suggested by \cite{case_study_guide}.

    A different kind of threat that considers the intrinsic quality of the construct is called \textit{poor construct definition}. In fact, if this is too narrow or too broad the results of the whole study are not reliable. Exploratory studies like this one marginally suffer this kind of problem since the construct definition is gradually built within the study itself.

    \textit{Mono method bias} happens when a single method of measurement is used and is applicable to the quantitative data that has been gathered in this study. Unluckily, given the time constraints that the Thesis impose i had to rely on a single method. However, given that the sole purpose of the analysis of the repositories is to validate the theory proposal crafted I believe that this is acceptable. Furthermore, I also used semi-structured interview for the same purpose.

\section{External Validity}
This validity threats consider if the findings of the study apply to other subjects/cases. However, for case studies the purpose is to enable analytical generalization in order to extend the results to cases with similar settings \cite{case_study_software_engineering}. Yin \cite{case_study_guide} states that this particular threat must be addressed when designing the study.

In this thesis, which follows the multiple, holistic case study, the cited literature suggests to make an extensive use of the theory in order to build robust foundations that, in turn, increase generalizability.


\section{Internal Validity}
Internal validity estimates the extent to which a casual relationships between variables is trustworthy and is achieved by removing or minimizing the bias. Although this threat type is more applicable to experiments or explanatory studies, there is one factor of interest for this thesis.

\textit{Selection bias} is relevant when sampling resources that are used to extract or validate the data. Therefore, it plays a central role when selecting assets for the theory validation phase. In particular, in order to obtain a sound validation it requires people with high expertise on the topic and that were involved in the development of the test suites that were mined as part of the study. For this reason the interviewees chosen are the five top contributors to such repositories.


\section{Reliability}
The reliability of a study refers to the ability of other researchers to replicate and verify the research results and its conclusions.
Whilst case studies have inherently poor reliability since they affect the context being studied during the study, this Thesis work is perceived to address this concern by thoroughly outlining the research design, metrics and procedures.
As such, the study's results should be possible to evaluate without replication and it is also perceived that the study should be replicable in another context given similar parameters to the described study.
