\chapter{Related work}

The reviewed related work presented in this section targets three aspects: 1) Broad studies targeting the concept of TD, 2) TD studies focusing on Testing, and 3) TD research related to Source Code. A special emphasys is used while addressing studies targeting both 2 and 3, since they represent rival theories and, hence, need a more in-depth analysis \cite{case_study_guide}. The last part of the chapter summarizes how this brief literature review has been conducted.

\section{TD Definition}
Identifying, estimating, tracking, and reducing TD and its impact are problems strongly tied with the definition of TD in use. Therefore, I believe it is necessary to address studies that shed light on such topic.

As stated before, the first ever definition of TD was provided two decades ago by Ward Cunningham \cite{first_mention_of_TD}. He defined the TD phonomenon as follows:
\textit{“Technical Debt includes those internal things that you choose
not to do now, but which will impede future development if left
undone. This includes deferred refactoring... Technical Debt doesn't include deferred functionality, except possibly in edge cases where delivered functionality is ‘good enough’ for the customer, but doesn't satisfy some standard (e.g., a UI element that isn't fully compliant with some UI standard).”}. Teudoropoulos et al.\ \cite{td_from_stakeholder_perspective} offer a similar view but with a more quality centric approach. They define TD as any gap in the technology infrastructure or its implementation that hinders the desired level of quality. However, this definition raise the problem of quantification of TD. In fact, their definition doesn't cover possible complex interaction and tradeoffs between quality attributes. %Furthermore, If we consider inversely proportional attributes it follows that TD is impossible to tame.
Kruchten et al.\ \cite{td_from_debt_to_metaphor} argue about following a more pragmatic definition of Technical Debt. They propose to exclude from the definition every aspect of the Software Lifecycle that is either completely visible (e.g.\ low external quality) or completely invisible (e.g.\ introducing a new functionality). This avoids overstimating the accumulated debt and, hence, allows a better response to market opportunities.

On the opposite end of the spectrum there are studies that, instead, try to broaden the umbrella of items which fall under the TD umbrella. Brown et al.\ \cite{td_current_vs_optimal_quality} define technical debt as the difference between the current state of the System and an \textit{hypothetical} ideal one. This broad description, however, is in clear contrast with the previous reported studies. In fact, following this specification it is possible to consider as TD also documentation (both technical and not), missing features etc. Furthermore, the definition is vague in terms of which stakeholder pictures the mentioned ideal system and, hence, TD quantification greatly varies depending on that. This problem implies that a universal or at least common approach for TD management is problematic if not impossible. Moreover, it does not account for incompatible points of view of stakeholders (e.g.\ developers' perspectives versus a managers' ones). A similar definition is proposed by Tom et al.\ \cite{exploration_of_td}. They extended a previous Systematic Literature Review \cite{slr,exploration_of_td_2} with a Multivocal Literature Review \cite{multivocal_literature_review} and crafted a broad but detailed definition of TD. It divides the phenomenon is sub-items that are easier to manage and, hence, allow a good level of separation of concerns while approaching TD. 
