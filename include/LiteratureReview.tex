\chapter{Related work}

The reviewed related work presented in this section targets two aspects: 1) Broad studies targeting the concept of TD, and 2) TD studies focusing on Testing TD. A special emphasys is used while addressing studies belonging in the latter category that also cover other Debt sub-dimensions \cite{mapping_study_td}, since they represent rival theories and, hence, need a more in-depth analysis \cite{case_study_guide}. The last part of the chapter summarizes how this brief literature review has been conducted.

\section{TD Definition} \label{td_definition}
Identifying, estimating, tracking, and reducing TD and its impact are problems strongly tied with the definition of TD in use. Therefore, I believe it is necessary to address studies that shed light on such topic.

As stated before, the first ever definition of TD was provided two decades ago by Ward Cunningham \cite{first_mention_of_TD}. He defined the TD phonomenon as follows:
\textit{“Technical Debt includes those internal things that you choose
not to do now, but which will impede future development if left
undone. This includes deferred refactoring... Technical Debt doesn't include deferred functionality, except possibly in edge cases where delivered functionality is ‘good enough’ for the customer, but doesn't satisfy some standard (e.g., a UI element that isn't fully compliant with some UI standard).”}. Teudoropoulos et al.\ \cite{td_from_stakeholder_perspective} offer a similar view but with a more quality centric approach. They define TD as any gap in the technology infrastructure or its implementation that hinders the desired level of quality. However, this definition raise the problem of quantification of TD. In fact, their definition doesn't cover possible complex interaction and tradeoffs between quality attributes. %Furthermore, If we consider inversely proportional attributes it follows that TD is impossible to tame.
Kruchten et al.\ \cite{td_from_debt_to_metaphor} argue about following a more pragmatic definition of Technical Debt. They propose to exclude from the definition every aspect of the Software Lifecycle that is either completely visible (e.g.\ low external quality) or completely invisible (e.g.\ introducing a new functionality). This avoids overstimating the accumulated debt and, hence, allows a better response to market opportunities.

On the opposite end of the spectrum there are studies that, instead, try to broaden the umbrella of items which fall under the TD umbrella. Brown et al.\ \cite{td_current_vs_optimal_quality} define technical debt as the difference between the current state of the System and an \textit{hypothetical} ideal one. This broad description, however, is in clear contrast with the previous reported studies. In fact, following this specification it is possible to consider as TD also documentation (both technical and not), missing features etc. Furthermore, the definition is vague in terms of which stakeholder pictures the mentioned ideal system and, hence, TD quantification greatly varies depending on that. This problem implies that a universal or at least common approach for TD management is problematic if not impossible. Moreover, it does not account for incompatible points of view of stakeholders (e.g.\ developers' perspectives versus a managers' ones). A similar definition is proposed by Tom et al.\ \cite{exploration_of_td}. They extended a previous Systematic Literature Review \cite{slr,exploration_of_td_2} with a Multivocal Literature Review \cite{multivocal_literature_review} and crafted a broad but detailed definition of TD. It provides a list of five area related to Software Development in which TD can accrue. Namely 1) code debt, 2) design and architectural debt, 3) environmental debt, 4) knowledge distribution and documentation debt, and 5) testing debt. Another important secondary study focusing on TD is \cite{mapping_study_td}. One of its Research Questions aims to refine and delimit the conpet of TD individuating its sub-components and boundaries. The authors propose a detailed list of ten dimensions which are ulteriorly divided by their cause.

The studies hereby include show the limit of the Technical Debt metaphor. Its success derives from the fact that it's easy to understand and to relate it with every-day experiences. However, this is also a drawback. In fact, diluting it is extremely easy and, hence, hinders a universal understanding of the phenomenon.

\section{Testing Technical Debt} \label{testing_td}

The term Testing Technical Debt refers to the TD that accrues in the context of \textit{testware}, meaning an umbrella term to identify artifacts and processes adopted while running tests. As indicated by \cite{mapping_study_td} there is a relevant number of sources targeting testing TD, but they focus mainly on lack of test automation, low code coverage, and lack of tests.  For instance \cite{test_automation_td} founds a possible source of Technical Debt in problems related to \textit{Test Automation}. As result of their case study they report four finding that cause testing technical debt to accrue. Namely: 1) \textit{Reuse and sharing of test tools brings issues that need to be considered}, 2) \textit{Test facility infrastructure is not transparent and may alter the test results if not accounted for}, 3) \textit{Generalist engineers expect that their tools are easy to use}, and 4) \textit{Accepted development practices for test code are potentially less rigorous than for production code}. As stated before, all of these use a environmental point of view as TD enabling factor. Also number 4, which might seem that target the intrinsic nature of test code, focuses on missing documentation (relative to test scripts) and quick hacks that are dependent on the system architecture and, hence, related to such TD dimension. Shah et. al.\ \cite{exploratorying_testing_td} studied manual testing as source of Technical Debt. Precisely Exploratory Testing. This technique, has the potential to greatly inflate the accumulation of technical debt since is, by nature, not reproduceable and, hence, not suitable for regression testing. The interest matured in using only this technique as mean of Quality Assurance is therefore extremely high.

Brwon et. al.\ \cite{td_current_vs_optimal_quality} consider not carrying out a test-plan the root cause of TTD since defects can remain and, hence, propagate to production code. 
A different approach is proposed by .... Their study shows a similarities between test code and source code...

However, after following the process highlighted in \ref{literature_review_process}, the number of studies showing similarities between test code and source code is small. Henceforth, there is a gap in the current state of the literature.

\section{Literature Review Process} \label{literature_review_process}