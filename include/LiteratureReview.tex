\chapter{Related work}

The reviewed related work presented in this section targets three aspects: 1) Broad studies targeting the concept of TD, 2) TD studies focusing on Testing, and 3) TD research related to Source Code. A special emphasys is used while addressing studies targeting both 2 and 3, since they represent rival theories and, hence, need a more in-depth analysis \cite{case_study_guide}.

\section{Broad Studies targeting TD}
Identifying, estimating, tracking, and reducing TD and its impact are problems strongly tied with the definition of TD in use. Therefore, I believe it is necessary to address studies that shed light on such topic.

Kruchten et al.\ \cite{td_from_debt_to_metaphor} argue about following a more pragmatic definition of Technical Debt. They propose to exclude from the definition every aspect of the Software Lifecycle that is either completely visible (e.g.\ low external quality) or completely invisible (e.g.\ introducing a new functionality). This avoids overstimating the accumulated debt and, hence, allows a better response to market opportunities.

