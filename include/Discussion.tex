\chapter{Discussion}

This section contains the discusses the results summarized in the previous Section and is divided per research question. Each technical debt item is described following the structure proposed by Li et al.\ \cite{mapping_study_td} indicating also which data was not applicable due to the nature of this study. Moreover, since one of the required fields in this proposed structure is the interest amount, this have been included while providing the description and hence section \ref{sec:res_rq3} provides possible explanation of this phenomenon.

\todo{rewrite this shit}

\section{Research Question 1}

	\subsection{Function complexity}
	
	
	\begin{table}[!htbp]
		\centering
		\tabulinesep=1.2mm
		\begin{tabu} to \textwidth {|X|X[3]|}
			
			\hline
			\textbf{Field} & \textbf{Value} \\ 
			\hline
			
			ID & Function complexity \\
			\hline
			
			Location & Logic of UI tests \\
			\hline
			
			Responsible & Tester developer \\
			\hline
			
			Type & Code Technical debt \\
			\hline	
			
			Description & The use of complex logic (e.g.\ deeply nested if-else statements) within the body of a test script.\\
			\hline
			
			%Date / Time & Not applicable \\
			%\hline
			
			%Principal & Non applicable \\
			%\hline
			
			Interest amount &  4/5 (High on Likert's scale) \\
			\hline
			
			%Interest probability & Not applicable \\
			%\hline
			
			%Interest standard deviation & Not applicable \\
			%\hline
			
			Correlations with other TD items & Dry Violations, God functions\\
			\hline 	 
			
			%Name & Not applicable \\
			%\hline 		
			
			%Context & Not applicable \\
			%\hline
			
			Propagation rules & It is very likely that it accrues since flattening a deeply nested hierarchy of statements is error-prone and tedious.\\
			\hline
			
			Intentionality & Intentional \\
			\hline 	 	
			
		\end{tabu}
		\label{tab:res-function-complexity}
		\caption[Function complexity specification]{Function complexity specification according to guidelines specified by \cite{mapping_study_td}.}
	\end{table}


\section{RQ2}

\section{Research question 3} \label{sec:res_rq3}


