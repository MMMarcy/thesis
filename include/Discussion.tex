\chapter{Discussion}

This section contains the discusses the results summarized in the previous Section on a per TD item basis, with a coarser clustering following the defined RQs. Each technical debt item is described following the structure proposed by Li et al.\ \cite{mapping_study_td}. However, fields not applicable to the nature of this study are omitted.

Finally, since one of the required fields in this proposed structure is the interest amount that is the raison d'etre of RQ3, the explanation is provided together with the description of every TD item. Therefore, a separate section for that RQ is omitted.

\section{Research Question 1}

    \subsection{DRY violations} \label{sec:res-dry-violations}
    
        DRY is an acronym that stands for Don't Repeat Yourself. Its essence is about that logic should be implemented once and only once and reused in different sections of the code-base when needed.

        The accretion of this TD item happens in two different circumstances. Firstly, when the code is poorly documented is harder for maintainers to know which primitives and functionalities are available, and the more extensive the code base, the more developer incur in this problem.
        
        The other source of DRY violations is the lack of tooling. In fact, developers and testers alike agree that not having a full-fledged IDE (Integrated Development Environment) causes these problems.
        
        In the Company, both causes are present with the lack of documentation being the predominant one. In fact, UFT provides an IDE that is considered inferior to other tools available for Source Code Development like Eclipse or IntelliJ IDEA. Therefore, part of this TD manifestations can be caused from this. However, the interviews revealed this to be more of a nuisance than a problem. On the other hand, the lack of documentation is perceived as a real problem: testers need to browse the source code to understand what a function does. Moreover, when asked how familiar they are with the functionalities provided by the common repository or other parts of the test code-base none of the interviewees provided an answer ascribable to a good knowledge.
        
        Moreover, the severity of this TD item is even greater if considered from a holistic point of view. In fact, duplicated logic inflates function complexity by requiring testers to reimplement such logic and the single responsibility violation, since if the logic is duplicated it means it can be extracted into a reusable entity that does that task.
        
        For these reasons, the author believes that this TD item perfectly matches the correspondent one in source code TD with no differences. Moreover, the interviewees perceive this problem as serious (4.5/5 on Likert's scale); roughly the same as expressed by Hunt and Thomas in \textit{The pragmatic programmer} \cite{thomas1999pragmatic}.
    
    
    	\begin{table}[!htbp]
		\centering
		\tabulinesep=1.2mm
		\begin{tabu} to \textwidth {|X|X[3]|}
			
			\hline
			\textbf{Field} & \textbf{Value} \\ 
			\hline
			
			ID & DRY Violations \\
			\hline
			
			Location & Logic of UI tests \\
			\hline
			
			Responsible & Tester developer \\
			\hline
			
			Type & Code Technical debt \\
			\hline	
			
			Description & Repeating the same logic of code in at least two different sections of the code-base\\
			\hline
			
			%Date / Time & Not applicable \\
			%\hline
			
			%Principal & Non applicable \\
			%\hline
			
			Interest amount &  4.5/5 (Between High and Very on Likert's scale) \\
			\hline
			
			%Interest probability & Not applicable \\
			%\hline
			
			%Interest standard deviation & Not applicable \\
			%\hline
			
			Correlations with other TD items & Functions Complexity, God functions\\
			\hline 	 
			
			%Name & Not applicable \\
			%\hline 		
			
			%Context & Not applicable \\
			%\hline
			
			Propagation rules & It is very likely that it accrues since flattening a deeply nested hierarchy of statements is error-prone and tedious.\\
			\hline
			
			Intentionality & Both intentional and non \\
			\hline 	 	
			
		\end{tabu}
		\label{tab:res-dry-violations}
		\caption[DRY violations specification]{DRY violations specification according to guidelines specified by \cite{mapping_study_td}.}
	\end{table}

	\subsection{Function complexity}
	
	
	The first TD item analyzed is Function Complexity as measured by the Cyclomatic Complexity metric described by McCabe \cite{cyclomatic_complexity}.

    This TD item accrues in the source code that contain the logic behind the UI tests. Specifically, within the functions that manipulate both the global state (e.g.\ the UI under test) and the data retrieved from it once an action occurs. A function, in the Software Engineering field, refers to a named section of the source code that perform a specific task and can use an input or solely rely on modification of the mentioned global state. The analysis performed in this study does not differentiate between those two variants since cyclomatic complexity applies to both. This metric captures how complex is the directed acyclic graph that links one statement to the other. Loops, IF-ELSE constructs, etc.\ inflate such complexity by spanning branches and loops that make harder to reason about the data flow.
    
    In the Company under study, there are people appointed as testers who are in charge of developing and maintaining such code base. These people are appointed as responsible since the test code health is under their responsibilities. Furthermore, the accretion of this TD item is also intentional, since it is possible to distinguish between a plain, easy to follow function and an intricate one. Moreover, by their admission, once the test code is deployed and effectively enters the testing pipeline, it becomes harder to remedy and refactor. In fact, \textit{"even if the code is messy refactoring it is hard because many functions start using the code, ..., and fixing it always breaks something else or introduce errors ..." }. This could explain the crescent trends recognizable in Figures \ref{fig:project_a_avg_complexity}, \ref{fig:project_b_avg_complexity}, \ref{fig:project_c_avg_complexity}. In fact, despite active efforts (e.g.\ the drop in complexity in series A1 noticeable after a refactoring effort), the average complexity increases revision after revision with few exceptions.
    
    \todo{Add the chart from antonio papaer about the drop crtical point model}
    
    For all these reasons, the author of this study believes that the available knowledge on Function Complexity related to Technical Debt is widely applicable in the realm of UI testing, with little differences. Most notably, these differences are about which role the code base has (e.g.\ regression tests or smoke tests) and the threshold above which a function is considered too complex. 
    
    The first item is necessary since smoke tests are, by nature, shallow and hence less impacted by changes in the product. Therefore, the interest paid in such tests is low. On the opposite side of the spectrum, regression tests are extremely sensitive and hence entail a high debt.
    
    The second difference is due the continuous manipulation of the global state that include both the test code and the application under test. This continuous referencing to external environments entails that most of the functions are referential opaque \cite{referential_transparency} and hence harder to decipher since variables are not directly visible to the developer.
    
    However, setting a definite threshold is out of scope for this study, and it is unlikely that such a value exists since it can be tool and testing language dependant.
    
    Finally, the enquiry revealed that such problem is inflated by the DRY violations described in the previous Section. In fact, as stated before, duplicate logic tends to appear within the body a function, and as expressed by one of the interviewee \textit{"we try to factor out the common parts when they appear in more than one place, but it is not easy. Especially when more than one tester maintains the code base ..."}.
        
    \todo{Enter relations with other TD items.}
    	
	
	\begin{table}[!htbp]
		\centering
		\tabulinesep=1.2mm
		\begin{tabu} to \textwidth {|X|X[3]|}
			
			\hline
			\textbf{Field} & \textbf{Value} \\ 
			\hline
			
			ID & Function complexity \\
			\hline
			
			Location & Logic of UI tests \\
			\hline
			
			Responsible & Tester developer \\
			\hline
			
			Type & Code Technical debt \\
			\hline	
			
			Description & The use of complex logic (e.g.\ deeply nested if-else statements) within the body of a test script.\\
			\hline
			
			%Date / Time & Not applicable \\
			%\hline
			
			%Principal & Non applicable \\
			%\hline
			
			Interest amount &  4/5 (High on Likert's scale) \\
			\hline
			
			%Interest probability & Not applicable \\
			%\hline
			
			%Interest standard deviation & Not applicable \\
			%\hline
			
			Correlations with other TD items & Dry Violations, God functions\\
			\hline 	 
			
			%Name & Not applicable \\
			%\hline 		
			
			%Context & Not applicable \\
			%\hline
			
			Propagation rules & It is very likely that it accrues since flattening a deeply nested hierarchy of statements is error-prone and tedious.\\
			\hline
			
			Intentionality & Intentional \\
			\hline 	 	
			
		\end{tabu}
		\label{tab:res-function-complexity}
		\caption[Function complexity specification]{Function complexity specification according to guidelines specified by \cite{mapping_study_td}.}
	\end{table}
	
	
	
    \subsection{Single responsibility violations}
    
    This TD item occurs when a tester fails do break down the test logic in reusable chunks and let the function perform more than one specific task. 

    The same considerations made for Section \ref{sec:res-dry-violations} apply with some differences. Specifically, even though a function embodies more than one task within its logic the Techincal Debt Interest yielded by this item is lower. In fact, according to the interviewees, it is possible that the logic that the tester failed to separate in a different function is indeed needed once in the whole codebase. This scenario results in worse code from a purely technical perspective, but it might be more efficient for the tester, and possibly it doesn't inflate the interest amount at all. According to the interviewees, this is the most typical situation, and hence they perceive this TD as low impact with respect to test maintenance. However, the medium score (3/5 on Likert's scale) in the interest amount field is due to the correlations between this TD item and others. Namely, DRY violations and Functions Complexity.
    
    Regarding the propagation rules, it is safe to assume that the accumulation of this TD item is likely. Firstly, it is challenging to define the boundaries of a function, i.e.\ what is the correct granularity. Secondly, identifying the violations and correct them is admittedly a tedious task. Finally, once the code-base is polluted with these violations, the accumulation ratio tends to increase. When implementing new logic developers tend to follow the flow of the existing code. That implies that when extending a high-quality codebase they put more efforts in the task, and the opposite happens when working with low quality one.
    
    For these reasons, the author believes that the intentionality of this TD item is both intentional and unintentional. During creation phase, i.e.\ when the code base is at the beginning of its life-cycle, the intentional part is preponderant. However, during maintenance, the unintentional aspect takes over.
        
    
    	\begin{table}[!htbp]
		\centering
		\tabulinesep=1.2mm
		\begin{tabu} to \textwidth {|X|X[3]|}
			
			\hline
			\textbf{Field} & \textbf{Value} \\ 
			\hline
			
			ID & Single responsibility violations. \\
			\hline
			
			Location & Logic of UI tests. \\
			\hline
			
			Responsible & Tester developer. \\
			\hline
			
			Type & Code Technical debt. \\
			\hline	
			
			Description & The creation and use of functions that perform more than one task.\\
			\hline
			
			%Date / Time & Not applicable \\
			%\hline
			
			%Principal & Non applicable \\
			%\hline
			
			Interest amount &  3/5 (Medium on Likert's scale). \\
			\hline
			
			%Interest probability & Not applicable \\
			%\hline
			
			%Interest standard deviation & Not applicable \\
			%\hline
			
			Correlations with other TD items & Dry Violations, Function complexity. \\
			\hline 	 
			
			%Name & Not applicable \\
			%\hline 		
			
			%Context & Not applicable \\
			%\hline
			
			Propagation rules & Likely. Both creation and maintenance of the code base are affected and without code revisions it is challenging to define boundaries for functions.\\
			\hline
			
			Intentionality & Both intentional and unintentional. \\
			\hline 	 	
			
		\end{tabu}
		\label{tab:res-single-responsibility}
		\caption[Single responsibility specification]{Single responsibility specification according to guidelines specified by \cite{mapping_study_td}.}
	\end{table}
	
	
	\subsection{Complex Statements}
	
	
	Assessing this Technical Debt item is controversial. In fact, some developers prefer to have code that develops horizontally whereas other vertically. However, among practitioners seem that the vertical flavor is the most widely adopted. For instance, the vast majority of IDEs (Integrated Development Environments) try to enforce developers to statements less than 80 characters long. The same is done by some code formatter and code linter. Some companies and open source organizations enforce every contributor to comply with their code styles (e.g.\ Google\footnote{\href{https://goo.gl/KSFfDO}{https://goo.gl/KSFfDO}} set an 80 character hard limit). The rationale is that short statements are more likely to perform one and only one action, and hence it is easier to follow the program's logic.

    As it is possible to see from figures \ref{fig:statement_complexity_project_a}, \ref{fig:statement_complexity_project_b}, \ref{fig:statement_complexity_project_c}, and \ref{fig:statement_complexity_project_d} the percentage of statements that are longer than the set limit tend to decrease over time in all but one repositories. Repository A1 matches the conceptual model highlighted by Martini et al.\ \inote{add Antonio's paper}. It is visible that the codebase has been refactored once the \textit{emergency limit} is reached. All the repositories in Project C and D show a decreasing trend. Perhaps because the testers in charge of these repositories tend to refactor the code at every possible occasion instead of creating considerable change in one submission. However, the reader should also notice the substantial difference in the range of complex statements. Repositories belonging to Project A and B are considerably lower than project C and D. This highlights the differences in style that different developers have. 
    
    Another interesting fact is that these results seem to show that a percentage of complex statements of 5\% or below is optimal. If fact, repository B1, which is constant below such value is stable over time. Moreover, repository A1 at every refactoring reaches the 5\% mark and then it increases again.
    
    Regarding the interest's amount, interviewees agreed that this is \textit{"not an issue"}. However, they also agreed that having a high number of complex statements hinders the possibility of seeing code repetitions. Therefore, the author believes that is possible to associate with this TD a medium amount of interest.
    
    Finally, this item is completely intentional, since every programming language allows to break the statement in sub-statements. A possible explanation is that while developing, people tend to write a statement that follows the broke-down solution of the problem they have in mind one action at the time. Therefore, it is easier to match a single statement with one of those abstract actions. Moreover, during this step developers focus more on achieving a solution than respecting high-quality code and hence they fail to format the code to adhere to the style guidelines.

	
	
	\begin{table}[!htbp]
		\centering
		\tabulinesep=1.2mm
		\begin{tabu} to \textwidth {|X|X[3]|}
			
			\hline
			\textbf{Field} & \textbf{Value} \\ 
			\hline
			
			ID & Comlex statements \\
			\hline
			
			Location & Logic of UI tests \\
			\hline
			
			Responsible & Tester \\
			\hline
			
			Type & Code Technical debt \\
			\hline	
			
			Description & An overuse of statements that perform more than one action at the time, both with nested constructs or piped. \\
			\hline
			
			%Date / Time & Not applicable \\
			%\hline
			
			%Principal & Non applicable \\
			%\hline
			
			Interest amount &  3/5 (Medium on Likert's scale). \\
			\hline
			
			%Interest probability & Not applicable \\
			%\hline
			
			%Interest standard deviation & Not applicable \\
			%\hline
			
			Correlations with other TD items & Dry Violations\\
			\hline 	 
			
			%Name & Not applicable \\
			%\hline 		
			
			%Context & Not applicable \\
			%\hline
			
			Propagation rules & Likely if no code formatting is enforced\\
			\hline
			
			Intentionality & Both intentional and unintentional. \\
			\hline 	 	
			
		\end{tabu}
		\label{tab:res-complex-statements}
		\caption[Complex statements TD item specification]{Complex statements Technical Debt item specification according to guidelines proposed by \cite{mapping_study_td}.}
	\end{table}


\section{RQ2}

\section{Research question 3} \label{sec:res_rq3}


