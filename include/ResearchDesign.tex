\chapter{Methods} \label{methods}
This section describes the processes used within this study. However, related Validity Threats are discussed in \ref{validity_threats}.

\section{Case Study Design}

\section{Data Gathering}
As recommended by Yin \cite{case_study_guide}, I gathered data from various sources in order to increase Triangulation and hence improve the study's reliability. Namely 1) Informal Interviews (a.k.a.\ Coffe Machine / Water Bowl Interviews), 2) Mining of Comapany's Forum, 3) Mining of Comapny's Issue Tracker, 4) Mining of Repositories, and 5) Semi-structured Interviews.

However, some of the data sources were used as secondary sources, in order to refine the inquiry conducted by other methods. I.e.\ I used Informal Interviews, and the mining of the forum mainly during the first phase of this Exploratory Case Study. This allowed me to conduct a more focused Inquiry on the problems emerged during this preliminary data gathering.

\subsection{Informal Interviews}
I have used this Data Source to overcome resource's scarcity. In fact, the I had to keep the number of Formal Interviews (see \ref{semi-structured_interviews}) as low as possible. Consequently,I made this choice to avoid using these resources during the preliminary phases of the research. \tocite{Reference a peper on such interviews}



\subsection{Mining of The Forum}

\subsection{Mining of The Issue Tracker}

\subsection{Analysis of The Repositories}

\subsection{Semi-Structured Interviews} \label{semi-structured_interviews}

\section{Interpretation of the Data}








\section{Literature Review Process} \label{literature_review_process}
The literature reviewed in this section comes from a systematic approach used to ensure that all relevant studies were considered during the review phase.

The process used consists in these six steps:
\begin{enumerate}
    \item Identification of secondary studies on the topic.\\
        This first stage of the review revealed that secondary studies on the topic are scarce. The search in the main databases for the field returned three hits: 1) a Sistematic Literature Review, 2) a complementary (with respect to 1) Multivocal Literature Review, and 3) a Systematic Mapping Study of Technical Debt.
        
    \item First level of snowballing and filtering by meta-data.\\
        From the analyzed studies in 1 I firstly considered the ones that in authors' opinion are not relevant for this Thesis. Specifically the ones targeting Testing Technical Debt, Source Code Technical Debt, or both. After this first coarse filter I eliminated the ones not relevant based on meta-data analysis (Title and Abstract). The remaining ones have been added to the list used during quality and relevance assessment. 
        
    \item Extension with main databases and filtering by meta-data.\\
        The main databases used for extending this literature review are: IEEE Xplore, Springer Verlink, Science Direct, and ACM. The keywords inserted were "Technical Debt" and "Testing Maintenance". The filtering process is analogous to the one used in point 2.
    
    
    \item Second level of snowballing and filtering by meta-data.\\
        From the studies remaining after point 2 and 3 I performed another round of snowballing. After that the papers included has been filtered with the same procedure used in 2 and 3.
    
    
    \item Assessing quality and relevance of studies.\\
        Finally, the list of studies remained have been analyzed throughoutly. The metrics used for evaluating the studies are relevance to this Thesis and intrisic quality of the study itself. 
    
    
    \item Inclusion.\\
        The studies left after step 5 were included in the literature review
    
\end{enumerate}





