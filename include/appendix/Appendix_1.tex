\chapter{Checklist for case studies} \label{checklist_for_case_studies}


{
\renewcommand*{\arraystretch}{1.5}
\begin{longtabu} to \textwidth {X[l]X[18]X[1]}
\caption[Checklist for case studies]{Runeson and H{\"o}st checklist for conducting a case study with links and/or answers.}\\
\multicolumn{3}{l}{\textbf{Case study design}}                                                                                              \\

1 & What is the case and its units of analysis?                                                                                          &  \\

2 & Are clear objectives, preliminary research questions, hypotheses (if any) defined in advance?                                        &  \\

3 & Is the theoretical basis — relation to existing literature or other cases — defined?                                                     &  \\

4 & Are the authors’ intentions with the research made clear?                                                                            &  \\

5 & Is the case adequately defined (size, domain, process, subjects…)?                                                                   &  \\

6 & Is a cause–effect relation under study? If yes, is it possible to distinguish the cause from other factors using the proposed design? &  \\

7 & Does the design involve data from multiple sources (data triangulation), using multiple methods (method triangulation)?               &  \\

8 & Is there a rationale behind the selection of subjects, roles, artifacts, viewpoints, etc.?                                              & \\

9 & Is the specified case relevant to validly address the research questions (construct validity)?                                          & \\

10 & Is the integrity of individuals/organizations taken into account?                                                                      & \\

\multicolumn{3}{l}{\textbf{Preparation for data collection}}                                                                                \\

11 & Is a case study protocol for data collection and analysis derived (what, why, how, when)? Are procedures for its update defined?       & \\

12 & Are multiple data sources and collection methods planned (triangulation)?                                                              & \\

13 & Are measurement instruments and procedures well defined (measurement definitions, interview questions)?                                & \\

14 & Are the planned methods and measurements sufficient to fulfill the objective of the study?                                             & \\

15 & Is the study design approved by a review board, and has informed consent obtained from individuals and organizations?                  & \\

\multicolumn{3}{l}{\textbf{Collecting evidence}}                                                                                            \\

16 & Is data collected according to the case study protocol?                                                                                & \\

17 & Is the observed phenomenon correctly implemented (e.g. to what extent is a design method under study actually used)? & \\

18 & Is data recorded to enable further analysis? & \\

19 & Are sensitive results identified (for individuals, the organization or the project)? & \\

20 & Are the data collection procedures well traceable? & \\

21 & Does the collected data provide ability to address the research question? & \\

\multicolumn{3}{l}{\textbf{Analysis of collected data}}                                                                                              \\

22 & Is the analysis methodology defined, including roles and review procedures? & \\

23 & Is a chain of evidence shown with traceable inferences from data to research questions and existing theory? & \\

24 & Are alternative perspectives and explanations used in the analysis? & \\

25 & Is a cause–effect relation under study? If yes, is it possible to distinguish the cause from other factors in the analysis? & \\

26 & Are there clear conclusions from the analysis, including recommendations for practice/further research? & \\

27 & Are threats to the validity analyzed in a systematic way and countermeasures taken? (Construct, internal, external, reliability) & \\

\multicolumn{3}{l}{\textbf{Reporting}}                                                                                              \\

28 & Are the case and its units of analysis adequately presented? & \\

29 & Are the objective, the research questions and corresponding answers reported? & \\

30 & Are related theory and hypotheses clearly reported? & \\

31 & Are the data collection procedures presented, with relevant motivation? & \\

32 & Is sufficient raw data presented (e.g. real life examples, quotations)? & \\

33 & Are the analysis procedures clearly reported? & \\

34 & Are threats to validity analyses reported along with countermeasures taken to reduce threats? & \\

35 & Are ethical issues reported openly (personal intentions, integrity issues, confidentiality) & \\

36 & Does the report contain conclusions, implications for practice and future research? & \\

37 & Does the report give a realistic and credible impression? & \\

38 & Is the report suitable for its audience, easy to read and well structured? & \\

\end{longtabu}
}